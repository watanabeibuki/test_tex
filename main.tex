\documentclass[uplatex]{jsarticle}
%\documentclass[a4j,10pt]{jarticle}

%%%%%%%%%%%%%%%%%%%%%%%%%%%%%%%%%%%%
\pagestyle{empty}
\sloppy\fussy
\setlength{\topmargin}{-20.4mm}
\setlength{\textheight}{271mm}
\setlength{\textwidth}{175mm}
%\setlen/Users/watanabeibuki/デスクトップ/LATEX/[.auxgth{\oddsidemargin}{-5.4mm}
\setlength{\evensidemargin}{-5.4mm}
\setlength{\headheight}{0mm}
\setlength{\footskip}{10mm}

\renewcommand{\baselinestretch}{0.92}
\renewcommand{\textfraction}{0}\renewcommand{\floatpagefraction}{1}
\renewcommand{\topfraction}{1} \renewcommand{\bottomfraction}{1}

\begin{document} 
    \twocolumn[
        \begin{center}
            {\Large 機械学習による論理素子配置の評価環境の構築}\\
            \vspace{0.1cm}
            {\large Constructing Evaluation Environment for Placement of Logic Device by Using Machine Learning}\\
            \vspace{0.2cm}
            {\large 1820232 渡邊 伊吹 2066008 讃岐 純平}\
            {\large 指導教員\ \  窪田 昌史, 弘中 哲夫}\\
        \end{center}
    ]
    \section{はじめに}
    災害現場などでドローンを活用した遭難者探索が導入され始めている.
    ドローンを活用することで,
    人の進入や長時間の探索が困難な場所での探索,
    広範囲の探索が可能となる.
    現在のドローンを活用した探索は,
    人の手による操縦と目視で検出対象の確認を行なっている.
    そのため,
    ドローンの操縦に技術を要されてしまうこと,
    目視による確認では見落としてしまう可能性がある
    といった問題があり,
    改善の余地があると考えられる.
    本研究では,ドローンの自動操縦と人の自動検出を行うシステムの開発の中で
    人の自動検出の部分における検出率の改善を行うことを目標とする.
\end{document}
%\title{タイトル名}
%\author{asdfghj}
%\date{\today}

%\begin{document}
    % \maketitle
    % Hello World!
%\end{document}