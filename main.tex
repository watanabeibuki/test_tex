\documentclass[uplatex]{jsarticle}
%\documentclass[a4j,10pt]{jarticle}
\usepackage[dvipdfmx]{graphicx}
%%%%%%%%%%%%%%%%%%%%%%%%%%%%%%%%%%%%
\pagestyle{empty}
\sloppy\fussy
\setlength{\topmargin}{-20.4mm}
\setlength{\textheight}{271mm}
\setlength{\textwidth}{175mm}
\setlength{\oddsidemargin}{-5.4mm}
\setlength{\evensidemargin}{-5.4mm}
\setlength{\headheight}{0mm}
\setlength{\footskip}{10mm}

\renewcommand{\baselinestretch}{0.92}
\renewcommand{\textfraction}{0}\renewcommand{\floatpagefraction}{1}
\renewcommand{\topfraction}{1} \renewcommand{\bottomfraction}{1}

\begin{document} 
    \twocolumn[
        \begin{center}
            {\Large 機械学習による論理素子配置の評価環境の構築}\\
            \vspace{0.1cm}
            {\large Constructing Evaluation Environment for Placement of Logic Device by Using Machine Learning}\\
            \vspace{0.2cm}
            {\large 1820232 渡邊 伊吹 2066008 讃岐 純平}\
            {\large 指導教員\ \  窪田 昌史, 弘中 哲夫}\\
        \end{center}
    ]
    \section{はじめに}
        FPGAなどの再構成可能デバイスの開発が進んでいく中で配置アルゴリズム
        についても研究が盛んに行われている.しかし、評価環境が均一でなく独自
        の評価で良しとしている現状がある.
        規模
        学習データ生成の面倒くささ、
        vprは高速処理が行える
        そこで本研究ではオープンソースCADであるVTRを用いることで
        一般的な評価環境が実現できるのではと考え、VTRを使用し
        評価環境を実現しようと思った.
        
        
    \section{環境の組み込み方法}
        VTRとはFPGAデバイス用のオープンソースCADをまとめたパッケージ
        でハードウェア記述言語(HDL)で記述された特定の回路を、
        指定のFPGAアーキテクチャにマッピングを行うものである.
        VTRの中には3つのオープンソースCADが入っておりそれぞれ
        OdinⅡ、ABC、VPRである。
        OdinⅡはVTRフローのHDLコンパイラで、論理合成に使用され、
        HDLをBLIF形式ネットリストに変換する.
        BLIF形式とはテキスト形式で論理レベルの階層回路を記述するための形式のこと
        ABCは論理最適化とテクノロジーマッピングを実行し、最適化するものである.
        VPRとはBLIF形式のネットリストを
        元にFPGAアーキテクチャに対して配置、配線を行うもの.
        このVPRは配置した結果に対して同じ形式で別の配置で置き換えても
        配線を行うことができる.





        
        % ドローンで撮影された画像は4056×3040のサイズで,
        % YOLOv3のニューラルネットワークの入力画像のサイズとして
        % 本システムでは640×480を使用している.
        % 入力画像のサイズの差は通常リサイズの画像処理によって調整されるが
        % この過程で一定の情報が失われる可能性がある.
        % 従来\cite{大澤佑哉}は,
        % 入力画像をあらかじめ4分割,16分割,64分割など4のべき乗での分割をしてから
        % ニューラルネットワークに投入する方法で
        % リサイズによる影響を低減する方法を採用していた.
        
        % 本研究ではYOLOv3のニューラルネットワークの入力サイズの640×480に
        % 従来よりも近くなるような
        % 分割方法で画像を投入する方法を提案する.
        % 図\ref{fig1}に示すように分割方法として縦に6分割,横に6分割の36分割で行うと,
        % 分割画像のサイズは676×506と676×507になり
        % 入力サイズの640×480に近づき,
        % リサイズによる影響をさらに低減することが期待できる.
    \begin{figure}[htbp]
        \begin{center}
            \includegraphics[width=7.3cm]{tekitounazu.png}
            \caption{配線結果}
            \label{fig1}
        \end{center}
    \end{figure}
    \section{配置配線の評価}
        VPRによって出力されるものには、ワイヤレングス、配線の成功or失敗、 配線回数、
        配線領域不足数、クリティカルパス、などがあり、これらを配置配線の評価と捉えることで
        配置の良し悪しの指標となる.




        % 物体検出の精度の評価指標として,mAP(mean Average Precision)を用いる.

        % 学習には,ドローンで撮影した328枚の画像を使用した.
        % 元の画像を分割して,人が写っている分割画像のみを抽出し,
        % 36分割の場合は1063枚の分割画像を使用した.
        % その内80%を訓練データ,20%をテストデータとした.
        % 学習には,CPU Intel Core i9 9900K 3.6GHz, メモリ32GB, GPU NVIDIA RTX3080のPCを使用した.
        % 学習回数は20000回で最高のmAPを調べた.
        % 提案手法の36分割では学習に約25時間かかった.
        % 表\ref{table}に実験結果をまとめた結果を示す.
        % 提案手法の36分割の画像(676×506,676×507)で学習を行なった場合,最高mAPは61.62%であった.
        % 従来の方法\cite{大澤佑哉}では
        % 元の画像(4056×3040)の場合,63.68%,
        % 4分割の画像(2028×1520)の場合,69.06%,
        % 16分割の画像(1014×760)の場合,68.25%,
        % 64分割の画像(507×380)の場合,54.82%
        % であった.

        % 今回の36分割の方法では物体検出の性能の改善が達成できていない.
        % 36分割の方法でも入力サイズの640×480に完全には一致しておらず,リサイズの処理を行う必要があり,
        % 情報の損失が発生していると考えられる.
        % 分割の際に,重なりを行いながら分割して,
        % 分割サイズを640×480に一致させる方法を採用すると
        % リサイズに影響をなくすことができる可能性がある.
        % \begin{table}[htbp]
        %     \begin{center}
        %     \caption{分割ごとの最高mAP}
        %     \includegraphics[width=7.5cm]{best_mAP.jpg}
        %     \label{table}
        %     \end{center}
        % \end{table}
    \section{まとめ}
        本研究によりVPRによる配線結果の出力から,配置結果の評価
        を行うことが可能になった.
        よってこれからは配線結果を基に自作配置ツールの学習データ採取を
        行い,機械学習の適用による配置評価を行っていきたい.
    \begin{thebibliography}{2}
        \small
        \bibitem{VTR8}
        Murray, O. Petelin, S. Zhong, J. M. Wang, M. ElDafrawy, J.-P. Legault, 
        E. Sha, A. G. Graham, J. Wu, M. J. P. Walker, H. Zeng, P. Patros, J. Luu, K. B. Kent and V. Betz “VTR 8:
        High Performance CAD and Customizable FPGA Architecture Modelling”, ACM TRETS, 2020
        % Joseph Redmon and Ali Farhadi, ``YOLOv3: An Incremental Improvement," CoRR, abs/1804.02767, 2018.
        % \bibitem{大澤佑哉}
        % 大澤佑哉,``自動人探索システムのためのYOLOv3を用いたドローン空撮画像からの人検出手法," 広島市立大学 情報科学部情報工学科 2019年度卒業論文, 2019.
        %\bibitem{AlexeyAB}
        %AlexeyAB版YOLOのURL
    \end{thebibliography}
\end{document}
%\title{タイトル名}
%\author{asdfghj}
%\date{\today}

%\begin{document}
    % \maketitle
    % Hello World!
%\end{document}